\chapter{Related Work\label{cha:chapter5}}

A significant amount of scientific work has been done on QAT. 
This research can be categorized based on different characteristics, 
which are covered separately in the following paragraphs.

\noindent
\textbf{Model architecture.}\\
RNN - \cite{ott2016rnn}
CNN - \cite{rastegari2016xnor}

\noindent
\textbf{Quantization target parameters.}\\
weights and activations - \cite{krishnamoorthi2018quantizing}\\
weights and activations - \cite{hubara2016qnn}\\
weights - \cite{polino2018modelcompression}
weigths - \cite{ott2016rnn}
weights and inputs - \cite{rastegari2016xnor}


\noindent
\textbf{Granularity of quantization.}

\noindent
\textbf{Handling of differentiability.}

\noindent
\textbf{Quantization precision.}\\
binary weights - \cite{courbariaux2015binaryconnect}
binary weights and activations - \cite{hubara2016qnn}
ternary weights - \cite{ott2016rnn}

\noindent
\textbf{Integration with pruning \& other techniques.}\\
pruning and Huffman Codign - \cite{han2016deepcompression}\\
distillation - \cite{polino2018modelcompression}

\noindent
\textbf{Modifications to loss functions.}
